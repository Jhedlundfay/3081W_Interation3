\hypertarget{index_User}{}\section{Manual for Non-\/\+Technical Users}\label{index_User}
Braitenberg vehicles are self-\/operating vehicles used to illustrate how simple functions can model complex behaviors. These vehicles are able to experience four distinct behaviors, fear, exploration, love, and aggression. The fear behavior causes robots to veer away from the objects their sensors are detecting which they fear. Likewise, the exploration behavior causes the robots to move away from an object that they have explored. The love and aggression behaviors cause the robot to pursue the object. The vehicles exist within an arena with light sources and food sources and can use their sensor to receive information about the food and light sources to decide how to move around within the arena. The robots are aggressive toward the food sources and either fear or explore the light sources (see the user interface instruction below). As time progresses within the arena, the robots become hungrier and, as they become hungrier, prioritize sensor readings from the food sources over those from the light sources. The simulation ends if any one of the robots starves (two minutes elapse in the simulator without the robot coming into contact with a food source).

The user interface, the buttons and sliders that appear by default on the left side of the arena, allow the robot, food sources, and light sources, to be manipulated so that the behaviors can be observed and compared for a variety of environments. The \char`\"{}\+New Game\char`\"{} button resets all of the robots, food sources, and light sources in the arena to new random positions and sizes. The pause/play button pauses the arena, during which time the arena will not update. If the simulation ends because one of the robots starves, a new game must be started and the pause/play button will be disabled. The robot slider allows the number of robots that will appear in the arena to be changed. Any number of robots can be placed within the arena between 0 and 10. The fear/exploratory ratio slider changes the number of robots that will experience the exploratory behavior toward the lights versus the number that will experience the fear behavior towards the lights. When the number is set to zero all of the robots will experience the exploratory behavior and when it is set to the maximum, all of the robots will experience the fear behavior. The food and light sliders allow the user to change the number of food or light sources that appear within the arena. The enable/disable food button overrides the food slider making it so the arena has no food sources but the robots also do not experience hunger. The light intensity slider modifies how strongly the robots will respond to the light sources in the arena. If the slider is set to zero the lights will appear black and the robots will not respond to the presence of the lights at all, conversely if the slider is set to its maximum value, the robots will respond strongly to the light sources. \hypertarget{index_User}{}\section{Manual for Non-\/\+Technical Users}\label{index_User}
The Braitenberg vehicle simulator centers around a model-\/viewer-\/controller pattern among the arena, graphics arena viewer, and controller. The graphics arena viewer is responsible for, upon receiving information about the arena from the controller, drawing the arena, the user interface, and all entities within the arena using the graphics libraries. The graphics arena viewer also processes any user inputs given through the user interface and passes them up to the controller to be applied to the arena. The controller conveys information from the graphics arena viewer to the arena and vice versa as necessary. To implement new sliders in the user interface, one must also create new methods in controller or else new commands in communication because nothing should pass directly from the graphics arena viewer to the arena.

All of the robots in the arena operate on a subject-\/observer pattern where the robots themselves receive no direct information about the other entities in the arena. All information about the arena is passed to the sensors directly from the arena, calculated into an impulse, and passed to the robot. The sensors are all identical, all that needs to be specified when they are initialized it what kind of entity types they are meant to recieve information from when the vector of entities is passed to them by the arena and which side of the robot they are on. All of the sensors for the robots are stored in a vector so that the passage of imuplses from the sensors to the motion handler can be done in a loop rather than having to be done individually for every sensor.

New behaviors for the robot should be added first to the constants in params and then added to the members for the robot. New behaviors are often done with respect to a given stimulus and are passed along the motion handler via the Handle\+Impulse method which uses a switch statement to decide which among the correct changes to the wheels\textquotesingle{} velocities should be applied.

New types of entities should be added by adding that entity type as a constant and then by adding that new type to the arena. Since the arena is responsible for creating entities and handling collisions between entities the entities must be created along with an appropriate creation method in entity factory and the cases for their interaction with the other existing types of objects in the arena shold be added to the update timestep method in the lower part of the method which pertains to when an entity should be updated with a collision and what information, if any, should be sent to those entities. 